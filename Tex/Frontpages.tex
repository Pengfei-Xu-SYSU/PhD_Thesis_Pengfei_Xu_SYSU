%---------------------------------------------------------------------------%
%->> 封面信息及生成
%---------------------------------------------------------------------------%
%-
%-> 中文封面信息
%-
\confidential{}% 密级:只有涉密论文才填写
\schoollogo{scale=0.6}{sysu_logo}% 校徽
\title{遗传算法优化的光子器件}% 论文中文题目
\author{许鹏飞}% 论文作者
\advisor{余思远~教授~中山大学}% 指导教师:姓名 专业技术职务 工作单位
\advisors{}% 指导老师附加信息 或 第二指导老师信息
\degree{博士}% 学位:学士、硕士、博士
\degreetype{工学}% 学位类别:理学、工学、工程、医学等
\major{光学工程}% 二级学科专业名称
\institute{中山大学电子与信息工程学院}% 院系名称
\date{2019~年~6~月}% 毕业日期:夏季为6月、冬季为12月
%-
%-> 英文封面信息
%-
\TITLE{Genetic Algorithm Optimized Photonic devices}% 论文英文题目
\AUTHOR{Pengfei Xu}% 论文作者
\ADVISOR{Supervisor: Professor Siyuan Yu}% 指导教师
\DEGREE{Doctor}% 学位:Bachelor, Master, Doctor。封面格式将根据英文学位名称自动切换,请确保拼写准确无误
\DEGREETYPE{Engineering}% 学位类别:Philosophy, Natural Science, Engineering, Economics, Agriculture 等
\MAJOR{Optical Engineering}% 二级学科专业名称
\INSTITUTE{School of Electronics and Information Technology, Sun Yat-sen University.}% 院系名称
\DATE{June, 2019}% 毕业日期:夏季为June、冬季为December
%-
%-> 生成封面
%-
\maketitle% 生成中文封面
\MAKETITLE% 生成英文封面
%-
%-> 作者声明
%-
\makedeclaration% 生成声明页
%-
%-> 中文摘要
%-
\intobmk\chapter*{摘\quad 要}% 显示在书签但不显示在目录
\setcounter{page}{1}% 开始页码
\pagenumbering{Roman}% 页码符号

随着计算机性能的不断发展和微加工工艺和精度的不断提升,人们意识到,可以摆脱常规的解析和经验方法的局限,在纯粹数值计算的更高自由度的舞台上进行光子器件设计。通过逆向思维进行物质场设计,如调整介质材料的结构和位置分布,进行精确的光场操控,可以得到需要的器件功能。通过纯粹数值计算的光子器件,可以在满足器件功能的情况下,极大的减小光子器件的尺寸,缩小芯片的面积并降低成本等,具有现实意义。

计算机的整体浮点计算性能呈“摩尔定律”式增长,然而由于微处理器的铜导线的物理带宽限制,微处理器的单核心性能已经出现难以逾越的瓶颈。纯粹粗暴的数值计算将难以满足尺寸越来越大、功能越来越复杂的电磁场求解。需要通过精妙、高效、低能耗、低时间周期的光子器件的设计和优化算法,来充分调用不断发展的分布式的性能资源。
本论文的主要工作,是通过在光子器件设计中引入遗传算法优化,实现了三种紧凑高效的光子器件。有一步刻蚀的下置硅光栅反射镜的氮化硅光栅耦合器、超材料硅波导的硅—氮化硅层间耦合器、以及基于超材料电磁响应的长通滤波器等。

1、一步刻蚀的下置硅光栅反射镜的氮化硅光栅耦合器:在借鉴下置硅光栅反射镜的氮化硅光栅耦合器的高性能方案下,通过加工复杂度和器件性能的权衡折中,提出了一种一步刻蚀的光栅耦合器设计方案。通过简单的光栅耦合器制备流程,实现了-2.5dB的耦合效率和65nm的1dB带宽的变迹光栅耦合器性能,以及-3.6dB耦合效率和70nm的1dB带宽的均匀光栅耦合器的性能。

2、超材料硅波导的硅—氮化硅层间耦合器:通过遗传算法优化,实现超材料硅波导与氮化硅波导之间的差拍层间耦合。还采用了创新的双层氮化硅波导,既解决了硅波导与氮化硅波导之间的高容差度层间耦合,也满足了足够大的层间间隔和足够小的层间串扰。最终实现了-0.6dB的层间耦合损耗,且1dB带宽覆盖了1530-1570nm的40nm波段。

3、基于超材料电磁响应的长通滤波器:受周期晶格超材料的米氏谐振的长通滤波特性的启发,通过遗传算法优化了一个紧凑的超材料的长通光子滤波器,在5×5$\mu$m2的器件面积下实现了长通滤波功能。而且还研究了通过器件版图缩放和器件级联,实现了长通滤波器过渡带波长偏移和过渡带斜率增强的特性。最终,在长波长1550nm处实现-0.5851dB的低插入损耗和短波长1450nm处>20dB的功率衰减。

通过遗传算法优化的上述三个光子器件,说明了遗传算法在光子集成器件设计中具有广泛的应用前景和潜力,在阿姆达尔定律推动下的多核心时代,遗传算法将在光子集成器件设计中发挥更加重要的作用。


\keywords{遗传算法,光子器件设计,光栅耦合器,层间耦合器,长通滤波器}% 中文关键词
%-
%-> 英文摘要
%-
\intobmk\chapter*{Abstract}% 显示在书签但不显示在目录

With the fast development of the computer performance and the precision of CMOS fabrication, people realize that, we can overcome the restrictions from traditional analytical and empirical photonic design methods. By numerical optimization methods, we can reversely design the material field, adjust the material structures and distributions to precisely control the electromagnetic field of light and achieve versatile functions. Numerically optimized photonic devices can not only fulfill the desired functions, but greatly reduce the device footprint, and then reduce the cost of the photonic chips and push forward the applications.

Guided by the Moore's Law, the computer flop rates had been improving exponentially in the past decades, however, due to the limitation of the copper wire bandwidth within integrated circuits, the microprocessors are facing an insurmountable bottleneck of clock rate and the single core performance. Simple and crude numerical algorithms are not competent for the larger footprint and complex electromagnetic computations. While a smart, efficient, low power and time consuming photonic device design algorithms are required to use the multi-core resources.

In this dissertation, the genetic algorithm is introduced into the photonic device design and three compact and high performance photonic devices are designed: one-step-etched silicon nitride grating coupler with beneath silicon grating reflector, metamaterial silicon waveguide interlayer coupler with double silicon nitride layers, and silicon metamaterial longpass filters :

1、the silicon nitride grating coupler with bottom silicon grating reflector scheme is proved to be excellent in improving the coupling performance. By balancing the performance and fabrication complexity, we proposed a one-step-etch grating coupler design, and can be fabricated by convenient fabrication process. And finally we achieved a apodized grating coupler with -2.5dB efficiency and 65nm 1dB bandwidth, and a uniform grating coupler with -3.6dB efficiency and 70nm 1dB bandwidth.

2、gradient index metamaterial silicon waveguide is employed to match with silicon nitride waveguide refractive index. By using a double layer of silicon nitride waveguide, not only ensure a robust silicon-silicon nitride interlayer coupling, but ensure a large silicon nitride and silicon layer spacing to reduce the crosstalk. And a metamaterial interlayer coupler is achieved with -0.6dB layer transition and the 1dB bandwidth cover the 1530-1570nm band.

3、 inspired by the longpass features of the Mie-resonance in the metamaterial rod lattices, a compact longpass filter is designed by genetic algorithms to achieve the on-chip photonic longpass filtering. And transition band shifting and roll-off enhancement features are also studied by device layout scaling and device cascading. And the longpass filter is fabricated and tested with insertion loss about -0.5851dB at longpass band 1550mn and a >20dB power attenuation at stopband 1450nm.

Three photonics devices are designed by genetic algorithm in this dissertation, indicating that the genetic algorithms are useful and universal methods in the photoinc device designing and optimizations. Moreover, in the multi-core era driven by Amdahl’d law, the genetic algorithm will gradually become a significant tool in the photonic device design and application.


\KEYWORDS{genetic algorithm, photonic devices, grating couplers, interlayer couplers, longpass filters}% 英文关键词
%---------------------------------------------------------------------------%
